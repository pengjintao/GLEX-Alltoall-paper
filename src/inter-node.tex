\section {RDMA-based Segmented inter-node all-to-all (RDMASa2a)}

For inter-node communication, our heuristic is that: "for a large message with size $m$, send it at once may slower than concurrently send two messages with size $\frac{m}{2}$".
Our idea is breaking a large all-to-all into several concurrent medium all-to-all. 

So, considering a multi-node all-to-all where one process on a node. RDMA-based RDMASa2a algorithm is shown in Algorithm \ref{RDMASa2a}:
\begin{algorithm}
\caption{Concurrent inter-node all-to-all}\label{RDMASa2a}
M: slice size \\
\SetKwProg{Def}{def}{:}{}
\Def{RDMASa2a(sendbuf,recvbuf,sz)} 
{
	SliceCount $\leftarrow$ max($\frac{sz}{M}$,1) \\
	StartShift $\leftarrow$ 0 \\
	Make sure all communication request finished. \\
	\For{i in range(0,SliceCount)}
	{
		len $\leftarrow$ M \\
		\If {i == SliceCount-1}{len $\leftarrow$ sz - StartShift}
		\For{j in range(0,p)}
		{
			target $\leftarrow$ (MyRank + i) mod procn \\
			\If{target == MyRank}{Copy Data Directly}
			\Else{
				Init a RDMA\_Request req\\
				\If{j == p - 1}{
					req.flag $|$= GLEX\_FENCE \\
					\If{i == SliceCount - 1}
					{
						req.flag $|$= LOCAL\_EVENT \\
						req.flag $|$= REMOTE\_EVENT
					}
				}
			}
			RDMA\_Send(req)
		}
	}
	Wait 2 events
}
\end{algorithm}
In Algorithm \ref{RDMASa2a}, $M$ represent a message size of a all-to-all segment. $M$ is usually a medium message.
In GLEX communication layer, there are three RDMA requests flag we need to introduce. GLEX\_FENCE, LOCAL\_EVENT and REMOTE\_EVENT.
A RDMA request with a GLEX\_FENCE flag means that the network card will start the new request after all previous requests finished.
A RDMA request with a LOCAL\_EVENT or REMOTE\_EVENT flag means that after this request finished, it produce a event at source/dest host.
Sender or receiver may poll the event queue to make sure the message is complete.
The overhead of GLEX\_FENCE is small but polling event queue will incur a lot of overhead.
So we set GLEX\_FENCE flag every all-to-all segment But only use once  LOCAL\_EVENT and REMOTE\_EVENT to notify host that the messages finished.
Compared to non-segmented all-to-all, RDMASa2a have the same event overhead.
For MPI-based pair-wise all-to-all, where for each message, there are n-1 event for n processes.
Compared to it RDMA-based pair-wise all-to-all or RDMASa2a has far less event overhead and avoid rendenzvous protocol overhead.
